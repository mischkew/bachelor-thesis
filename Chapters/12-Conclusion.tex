\documentclass[../ClassicThesis.tex]{subfiles}
\begin{document}

%************************************************
\chapter{Conclusion and Future Work / Klara}\label{ch:conclusion}
%************************************************

\section{Lasercutting 3D-models}
% ultimate goal
People have always come up with brilliant ideas. Currently, the way people fabricate their prototypes is changing. With a 3D printer you can create objects without having any knowledge on the fabrication process. All it takes is the push of a button.\\
Thanks to the wide popularity of 3D printing there are a lot of 3D models available on the internet.\\
We have shown that such models can be converted to a 2D plan which can be lasercut. Originally, one needed a lot of know how for creating such plans. The transition from the 3D world to a flat surface is not an easy task. Especially since the flat objects need connectors which have to fit into the object when assembled. \\
Our software automatically generates a 2D plan for lasercutting a 3D model. This supports the progress of automation of fabrication techniques. 

\section{Maker Faire Ruhr, Vienna, Hanover}
For receiving feedback on our application we participated in three Maker Faires. \\
Those are events where people who have had an idea came up with a solution or a prototype and want to exhibit it to share their learnings and obtain feedback.\\
Visitors at our booth were very interested in our work. Many told us they built or bought a lasercutter for their home. But most of them said they never thought of making such 3 dimensional objects which we presented to them. 
\\\*\\
We want to open up new possibilites to have anyone, even without a lot experience with a lasercutter, be able to put their ideas into practice.


\end{document}