\documentclass[../ClassicThesis.tex]{subfiles}
\begin{document}

% ************************************************
\chapter{Introduction}
\label{ch:introduction}
% ************************************************

\section{Introduction}

% people are awesome -> ideas
% bringing ideas to live -> back in time -> difficult
% transition, changes now
% machines -> plug into your computer -> take over fabrication for you


People always had ingenious ideas. Though, back in time it was
difficult bringing these ideas to live. For example in the field of
engineering, when one has an idea how airplanes fly better
(Figure~\ref{fig:intro-ideas:plane}). Or ideas concerning an creative
process, like a fashion line for shoes which offer correct fit to the
shape of the feet of its wearer (Figure~\ref{fig:intro-ideas:shoe}).
Even in health care broad ideas evolve. In 2013 Jake Evill presented
\name{Cortex}, a customizable plaster that is \enquote{fully
  ventilated, super light, shower friendly, hygienic, recyclable and
  stylish}\footnote{\url{http://www.evilldesign.com/cortex}}, see
Figure~\ref{fig:intro-ideas:plaster}.

The way ideas are turned into reality is changing. Today, machines
exist that we can plug into our computer that will take over the
fabrication of objects. Thus, we can fully concentrate on the ideas
instead of requiring and acquiring additional skills to actually
fabricate.

% one of theses machines => 3d printer
% market -> growing, machines more and more affordable
% layerwise plastik materials, free forms, easy to use -> button click -> object out.
% but: limitations -> slow, material (lose structure/ beauty)

Such a fabrication machine is a 3D printer. 3D printing is an emerging
technology spreading across the consumer market. According to
\name{Wohlers Associates}, in the year 2015 the 3D printing market was
worth 6.5 billion USD. Their prognosis estimates a roughly 300\%
growth in the next five years\cite{wohlers-market}. With 3D printers
literally any user can produce any free-form object with a button
press. A common technique in 3D printing is fused deposition modeling
(FDM)\cite{}\note{source that FDM is common}. With the FDM technique,
thermoplastic material is heated and then pressed through a nozzle,
mounted on a print head. Thus, layer by layer material is extruded to
create three dimensional objects\cite{}\note{source thats how FDM
  works}. Though 3D printing brings a substantial progress in the
field of fabrication, it shows two flaws: 3D printing is slow and 3D
printing is limited in its materials. Even small models need a couple
of hours to be printed. The miniature bird house in
Figure~\ref{}\note{figure} printed for three hours. In its original
dimensions it needs about two days of printing time. The FDM technique
requires material to be pushed through a nozzle. Thus, the material is
melted and loses its original structure and characteristics.

% other machine, our target machine -> laser cutter
% cutting materials from a cut plan (typically svg file)
% 2d shapes/ plates
% assembled to objects
% materials like wood, acryllic, textiles, metal or stone
% laser cutter -> low-fi fabrications
% faster than printer
% market laser cutters is coming

Another fabrication machine is the laser cutter, shown in
Figure~\ref{}\note{figure}. A laser traces the outlines of a planar
cutting plan, so actual plates are cut from the materials
(Figure~\ref{}\note{figure}). Such materials are wood or acrylic.
High-performance devices can cut textiles, metals or stone. The laser
cutter is fast and preserves the original structure of the used
materials. Figure~\ref{} and Figure~\ref{} show how the cut-out plates
are assembled to a three dimensional object in several minutes. With a
laser cutter, we perform low fidelity fabrications. Low-fi means, we
produce functional objects that fulfill their purpose, but lack the
amount of detail a full-fledged product would have. Nonetheless, we
can iterate several times a day on our object to produce better
results. In comparison to 3D printing, we can use the time more
efficiently. Such a functional object is a quadcopter produced from
wood, shown in Figure~\ref{}\note{figure}. With a laser cutter, we can
also produce decorative objects (Figure~\ref{}\note{figure}). Here, we
benefit from the original characteristics of the materials.

When creating three dimensional objects with a laser cutter, users
face two challenges: They need a vast amount of spatial orientation
and they need craft men's knowledge about the used materials.


Naturally, it has other flaws.


% you
% can build 3d things from the laser cutter as well. though not
% freeform, but it is possible iterating on functional objects -> drone
% original materials can be used -> designer glasses but 2 challenges:

% need to assemble. spatial orientation. need to know where plates from the cutting plan will resemble on final model.
% need knowledge about materials. knowledge a carpenter or craftsman has. fingerjoints for example
% manual drafting of cutting plans for large objects -> difficult
% each connection must be placed exactly, on millimeter, others plates stuck or fall apart

% we aim to make 3d model fabrication with laser cutter easy. using its pros in time, material. improving the creation of cut plans
% we present software system platener
% similar intuitive approach as with 3d printing. start from 3d model. user knows how results will look lik.
% our software converts 3d model to a suitable cutting plan.
% analyzes geometries -> approximating a model with plates (different techniques we elaborate on in this thesis)
% putting plates together (software knows which materials -> allowing to place finger joints calibrated for the laser cutting, allowing bends).

% in this thesis we talk about:

% architecure
% data structures
% approximation of high-res models
% plate finding
% plate joining
% advanced analysis and conversion technique (future work)

% \section{Related Work}


% The work presented in this paper builds on personal fabrication,
% low-fidelity fabrication, and shape approximation.

% Personal Fabrication and Laser Cutting

% Laser-cutters are used for a wide variety of objects in personal
% fabrication as they allow users to quickly fabricate
% objects that fulfill a technical function: Coros et al. [2]
% show how to fabricate animated mechanical toys using
% laser cutters, SketchChair [19] shows how to build functional
% furniture, and Pteromys [21] shows how to laser cut
% functional model airplanes.

% Low-fidelity fabrication

% Platener shares the same goal as previous low-fidelity fabrication
% system, which aim at reducing fabrication time by
% replacing the original 3D representation with an alternative
% one: faBrickator replaces sub-volumes of a 3D model with
% Lego bricks [17]. LaserOrigami [16] approximates 3D
% shape by bending laser-cut 2D using a defocused laser.
% WirePrint replaces the surface of a 3D model with a
% wireframe structure [15].

% Converting 3D models into stackable 2D plates

% The most common approach to represent a 3D model using
% laser-cuttable parts is to convert it to a stack of 2D plates.
% This stack is generated by rasterizing/slicing the model
% along its vertical axis at a fixed interval. This functionality
% is included in many commercial products, such as the 3D
% editor 123D Make [1]. Hildebrand et al. [7] take this ap-
% proach one step further by slicing parts of the model in
% different orientations to better approximate the overall
% shape. While stackable plates ensure that the volume of a
% 3D object is well approximated, they are a less accurate
% surface representation.

% Converting 3D models into intersecting 2D pieces

% A different representation of a 3D model can be achieved
% using intersecting planar pieces. The resulting objects require
% less material and consist of fewer parts and thus are
% easier to assemble. SketchChair [19], for instance, provides
% a user interface for creating chairs based on intersecting
% pieces. McCrae et al. [6] and Schwartzburg et al. [20] provide
% algorithms that guarantee the constructability of designs
% based on intersecting pieces. Slices [12], finally,
% minimizes the number of pieces based on a study of human
% perception of shape. For some classes of objects, such as
% many types of furniture, a representation based on intersecting
% pieces equals the original shape (e.g., Furniture
% Factory [18] and Lau et al. [9]). Instead of converting an
% existing 3D model to intersecting pieces, FlatFitFab [13]
% allows users to create 3D models directly from lasercuttable
% parts.

% Converting 3D models to foldable 2D sheets

% To better represent the shape of a 3D model, several systems
% have been developed to convert 3D models into planar
% sheets that can be folded into the desired 3D shape. Mitani
% et al. [14] propose an algorithm that unfolds a 3D model
% into a set of strips that can be cut out of paper. Besides
% using hard creases to fold material into 3D shapes, Curved
% Folding [8] proposes to integrate smooth curves in the
% bending pattern. Soft folding [22] combines both approaches
% and allows users to include creases and soft folds into a
% flat sheet. While foldable sheets result in an accurate surface
% representation, they have no infill and are thus not
% sturdy enough to perform mechanical functions.
% Combining different fabrication techniques
% In the context of architecture, Martens et al. [11] manually
% segment a 3D model into both laser-cut and 3D printing
% parts to make use of the benefits of both tools. Platener
% facilitates combining different fabrication techniques by
% automating the substitution

% \subsection{Low-Fi Fabrication}


% \paragraph{Low-Fi Fabrication}

% Low-fidelity fabrication

% Platener shares the same goal as previous low-fidelity fabrication
% system, which aim at reducing fabrication time by
% replacing the original 3D representation with an alternative
% one: faBrickator replaces sub-volumes of a 3D model with
% Lego bricks [17]. LaserOrigami [16] approximates 3D
% shape by bending laser-cut 2D using a defocused laser.
% WirePrint replaces the surface of a 3D model with a
% wireframe structure [15].

% \paragraph{Platener}

% \paragraph{faBrickation}

% \paragraph{brickify}

% \paragraph{WirePrint}


% \subsection{Laser Cutters and {\threedmodel}s}

% \paragraph{LaserOrigami}

% \paragraph{SketchChair}

% \paragraph{123Make and Plate Stacking}

% \paragraph{FlatFitFab}


\end{document}

%%% Local Variables:
%%% mode: latex
%%% TeX-master: "../ClassicThesis"
%%% TeX-command-extra-options: "-shell-escape"
%%% End:
