\documentclass[../ClassicThesis.tex]{subfiles}
\begin{document}

%************************************************
\chapter{Implementation Goals and Toolchain}\label{ch:toolchain}
%************************************************

\newcommand\myNotes[1]{\textcolor{red}{#1}}

\section{Deployed as a Web-Service}

\subsection{Cross-Platform Browser Application}

\begin{itemize}
\item web service
\item cross platform
\item no installation necessary
\item target group: makers, run in labspaces, everywhere
\item typical HTML, CSS, Javascript trio
\end{itemize}

\subsection{Headless Version for Integration with Other Software}

\begin{itemize}
\item same service without visuals
\item batch processing or single
\item idea: integrate with other 3d editing applications to benefit
  from platener conversions
\item running on nodejs v 012
\end{itemize}

\section{Technologies and Libraries}


We use the Web Graphics Library (WebGL), because we want our
rendering results to have sophisticated visual effects. With
the use of WebGL we can process data on the GPU and execute
special effects code (shader code) in
web-browsers~\cite{mdn-webgl}.

On top of WebGL we use
{\threejs}\footnote{\url{http://threejs.org}}. {\threejs} is
a {\javascript} library, which wraps the WebGL API to make
the usage of WebGL simpler. It provides a scene graph
implementation. {\threejs} data representations are used
throughout {\convertify}.


\begin{itemize}
\item web service -> necessarily javascript
\item coffeescript preprocessing, no es6 because previous code base
  brickify was written in coffeescript -> save effort of rewrites
\item webgl -> rendering, now almost all major browser support it
  \myNotes{prove that}
\item nodejs for headless version/ server (backend) support (v0.12
  because old brickify dependencies)
\item threejs -> most acknowledged 3d web library (scene graphs and
  webgl)
\item bluebird -> promises (nice tech in js to handle async behavior,
  better than natives because of error handling, more features)
\item polyfills -> more cross platform support and isomorphic =>
  browsers and nodejs run in different vms
\item styles with stylus lib (also taken from brickify)
\end{itemize}

\section{Build System and Development Setup}

\begin{itemize}
\item gruntfile -> and webpack
\item different setups for client, server and cli in prod/ dev mode
\item test runner using mocha
\item bundle code into minified code and output sourcemaps (nice for
  in browser debugging)
\item smart server (rebuilding on filechanges, hotreloading (not
  compiling everything again for speedup))
\item browser lifereload (code change immediatly displayed in browser)
\item react hotreloading -> display changes in ui even without a
  browser reload (super fast feedback loop)
\end{itemize}

\end{document}

%%% Local Variables:
%%% mode: latex
%%% TeX-master: "../ClassicThesis"
%%% TeX-command-extra-options: "-shell-escape"
%%% End:
